\documentclass[12pt]{article} %[font size, tamano de hoja]{Tipo de documento}
\usepackage[a4paper]{geometry}
\usepackage[utf8]{inputenc} %Formato de codificación
\usepackage{csquotes}
\usepackage{amsmath}
\usepackage[spanish, es-tabla, es-nodecimaldot]{babel}
\usepackage{float} %Para posicionar figuras
\usepackage{graphicx} %Para poder poner figuras
\usepackage{hyperref} %Permite usar hipervínculos 
\usepackage{multicol} %Para hacer doble columna
\usepackage{caption}
\usepackage{verbatim}
\graphicspath{{images/}}
\usepackage{biblatex}
\usepackage{csquotes}
% \usepackage[style=numeric]{biblatex}
% \addbibresource{main.bib} %Import the bibliography file

\title{Evaluación del desempeño de interpolaciones y cálculo de trayectorias}
\author{Tomás Benavidez, Franco Amato\\ [2mm] %\\ para nueva línea
\small Universidad de San Andrés, Buenos Aires, Argentina}
\date{2024}

\begin{document}

\maketitle

\begin{abstract}
    El presente trabajo aborda la evaluación del desempeño de diversos métodos de interpolación de funciones y cálculo de trayectorias, explorando métodos clásicos como la interpolación de Lagrange y Splines cúbicos. Además, se analiza el impacto del número y la distribución de los puntos de interpolación en la precisión de los resultados. Se emplea un criterio de error absoluto para comparar la calidad de las interpolaciones frente a las funciones y trayectorias reales, mostrando que la interpolación con Splines cúbicos tiende a ajustarse mejor en diversos escenarios y que el número y método para elegir los puntos para interpolar tiene un gran peso. Se concluye resaltando la importancia y los desafíos de llevar la teoría a la práctica en aplicaciones numéricas complejas en el ámbito de la ingeniería y la ciencia computacional.
\end{abstract}

\vspace{0.1cm}

\begin{multicols}{2}
\raggedcolumns

\section{Introducción}

\section{Métodos preexistentes}

\subsection*{Metodo de aproximación de polinomios Runge-Kutta Orden 4}

\subsection*{Metodo de Euler}

\section{Procedimientos y resultados}

\subsection{Análisis del crecimiento de una especie en un sistema cerrado}\label{sec:1}
\subsubsection{Modelos de Crecimiento}\label{ssec:1a}

Primero, se estudió el desempeño de dos modelos de crecimiento: el modelo Exponencial y el modelo Logaritmico. Para el primero, dada una especie N, un crecimiento en el tiempo N(t), 
y una tasa instantanea de crecimiento r, podemos describir el crecimiento como 

% Aquí deberías incluir la ecuación

\section{Conclusiones}

\begin{thebibliography}{99} % 99 es un espacio reservado para el ancho del número de referencia más grande

\end{thebibliography}

\end{multicols}

\end{document}